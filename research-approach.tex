\chapter{RESEARCH APPROACH}
\section{PROV-PUB-O: a provenance ontology for research publications}
This section presents the approach to the creation of an ontology for capturing provenance in the process of preparing research publications. The ontology is divided into two parts. 

The first part describes the structure of a research publication. We base our work on the Document Components Ontology (DoCO), which is one of the Semantic Publishing and Referencing Ontologies (SPAR). SPAR focuses a lot on modeling citations and bibliographic references. Four out of the eight ontologies in SPAR deal with citations and bibliographic references. (They are CiTO, the Citation Typing Ontology, FaBiO, the FRBR-aligned Bibliographic Ontology, BiRO, the Bibliographic Reference Ontology, and C4O, the Citation Counting and Context Characterization Ontology.) So SPAR is good for modeling the linkage between publications. In this thesis we focus on the reproducibility of a certain publication, so we do not delve deep into citations and bibliographic references as they have nothing to do with the computational experiments presented in the current publication. We model the structure of a research publication because experimental results must reside somewhere in the publication. 


\section{Automatic provenance capturing mechanism} 
\cite{freire2014reproducibility}
\cite{groth2006architecture}


\section{A proof-of-concept platform}

%\resetfootnote %this command starts footnote numbering with 1 again.

%\begin{figure}
%\centering
%\vspace{2.0in}
%\caption[A Shorter Caption for the List of Figures]
%   {This is the Caption for the First Figure in Chapter 2.  It is a
%    long, long caption; we do not want to put the whole thing in the
%    List of Figures. A Shorter Caption can go in the square brackets.}
% If you like additional lines in the caption indented, see the root template
% file rpithes.tex for an example of using the caption package to do this.
%\end{figure}
 
%This is shown in table~\ref{mytable}.  % see \label below
 
%\begin{table}
%\caption{This is the Caption for Table 2}
%\label{mytable}        % \label command must always comes AFTER the caption
%\begin{center}
%\begin{tabular}{lll}
%Here's       & another     & example  \\
%of           & a           & table    \\
%floated      & with        & the      \\
%\verb+table+ & environment & command.
%\end{tabular}
%\end{center}
%\end{table}


%\section{This is a Section Heading}
 
%\subsection{This is a Subsection Heading} 
 
%Text before a footnote.\footnote{Here's the text of the footnote.}
%Text after the footnote.


%%% Local Variables: 
%%% mode: latex
%%% TeX-master: t
%%% End: 
