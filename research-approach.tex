\chapter{RESEARCH APPROACH}
\section{PROV-PUB-O: a provenance ontology for research publications}
This section presents the approach to the creation of an ontology for capturing provenance in the process of preparing research publications. The ontology is divided into two parts. 

The first part --- Section~\ref{subsec:structure} --- describes the structure of a research publication. We base our work on the Document Components Ontology (DoCO), which is one of the Semantic Publishing and Referencing Ontologies (SPAR). SPAR focuses a lot on modeling citations and bibliographic references. Four out of the eight ontologies in SPAR deal with citations and bibliographic references. (They are CiTO, the Citation Typing Ontology, FaBiO, the FRBR-aligned Bibliographic Ontology, BiRO, the Bibliographic Reference Ontology, and C4O, the Citation Counting and Context Characterization Ontology.) So SPAR is good for modeling the linkage between publications. 

In this thesis we focus on the reproducibility of a certain publication, so we do not delve deep into citations and bibliographic references as they have nothing to do with the computational experiments presented in the current publication. 

We model the structure of a research publication because experimental results must reside somewhere in the publication. Textual elements in the publication provide contextual information and explanation for the reported results. Although these elements are not parts of the process leading to the reported results, they are documentation that help readers to correctly interpret the results, so we include structural elements of research publications in the provenance.

The second part --- Section~\ref{subsec:process} --- describes the process leading to the reported results. It is a specialization of the W3C Provenance Ontology (PROV-O) for the use case of research publication preparation. We base the development of PROV-PUB-O on PROV-O because using general provenance ontologies such as PROV-O proves to be an effective way to keep track of the lineage of the source data and the changing processes leading to the final results. 

The goal of the specialization is to make the specialized ontology, namely PROV-PUB-O, detailed enough to enable the creation of \emph{executable provenance graphs} (EPGs). In an EPG, the semantics of each element in the process leading to the reported results is well defined, so the replication of the process becomes straightforward. Moreover, the process described with an EPG can even be used to validate the scientific conclusions by allowing readers to adapt the experiment reported in the paper and carry out their own studies.

The hypothesis to justify for PROV-PUB-O is that it is useful in the use case of modeling research publication preparation process in the domain of earth science for the purpose of replicating data transformation processes and validating scientific conclusions. We would like to show that less effort is required to create EPGs with PROV-PUB-O than with PROV-O. We will cover the measurement of effort amount and the details of the benchmark modeling tasks in Section~\ref{subsec:evaluation}.

\subsection{Publication structure modeling}
\label{subsec:structure}
Document structure models such as the Document Components Ontology (DoCO)\footnote{DoCO, the Document Components Ontology: http://purl.org/spar/doco} are usually based on structural patterns \cite{di2014dealing} and rhetorical structure theory (RST) \cite{taboada2006rhetorical}.

Here we take DoCO as an example. DoCO defines structural (e.g. block, inline, paragraph, section, chapter, etc. --- meaning what the component looks like) as well as rhetorical (e.g. introduction, discussion, acknowledgements, reference list, figure, appendix, etc. --- meaning what role the component plays in the document) document components.

Figure~\ref{fig:doco} shows the components of DoCO and the respective classes included in these components.
\begin{figure}[h]
	\includegraphics[width=\textwidth]{doco-architecture.png}
	\caption{Architecture of DoCO, the Document Components Ontology}
	\label{fig:doco}
\end{figure}

Structural document component types are closely related to the typesetting of the component contents. For example, if a block of text consisting of a title and several paragraphs is known to be a chapter, then at the time of rendering this block, the title can be made larger, proper font and size can be set for the content paragraphs, and the proper spacing can be set between this block and the blocks previous and next to it. Therefore, each structural document component type is like a class used to label HTML elements, and to be rendered according to class styles defined in Cascading Style Sheets (CSS).

Rhetorical document component types are completely irrelevant to typesetting --- just knowing a block of text is an ``introduction'' has nothing to do with properly formatting it. Rather, rhetorical types focus on the \emph{relations} between document parts, so these types are actually \emph{types of relations}. For example, the ``introduction'' component of a paper is usually a chapter, but theoretically it can be a section or even just a paragraph, leading to very different typesetting options. But just look at how we call this ``introduction'' component -- we call it ``introduction \emph{of} a paper'', i.e., an introduction component is always \emph{of something}. It does not make sense to have a ``floating'' introduction attached to nothing. 

In light of this relational perspective of rhetorical types, we make the following changes from DoCO:

\begin{itemize}
	\item The sro:Abstract class is changed to the pub:abstract property, whose domain is (doco:Chapter or doco:Section) and (dcterms:isPartOf some doco:BodyMatter or doco:FrontMatter), and range is pub:Document. The prefix sro expands to http://salt.semanticauthoring.org/ontologies/sro\# and stands for SALT (Semantically Annotated \LaTeX \cite{groza2007salt}) Rhetorical Ontology. The prefix doco expands to http://purl.org/spar/doco/, and the prefix dcterms expands to http://purl.org/dc/terms/, meaning DCMI (Dublin Core$^{\textregistered}$ Metadata Initiative) Metadata Terms\footnote{DCMI Metadata Terms: http://dublincore.org/documents/dcmi-terms/}. Finally, the prefix pub is for PROV-PUB-O, which is under development and not published yet.
	\item The sro:Background class is changed to the pub:background property, whose domain is (doco:Chapter or doco:Section) and (dcterms:isPartOf some doco:BodyMatter), and range is pub:Document. 
	\item Classes sro:Conclusion, sro:Contribution, sro:Discussion, sro:Evaluation, sro:Motivation, sro:Scenario are changed in the same manner as sro:Background. Note that these classes are different from sro:Abstract in that they cannot be part of the front matter component of a document.
	\item The class deo:Acknowledgements \comment{not sure yet.} The prefix deo here expands to http://purl.org/spar/deo/ and stands for the Discourse Elements Ontology\footnote{The Discourse Elements Ontology: http://purl.org/spar/deo}.
\end{itemize}

The hypotheses to justify here is that properties are more useful than rhetorical concepts in DoCO for describing publication components. We plan to conduct user survey to collect opinions about pairs of RDF statement groups such as

\begin{quote}
	\begin{tabular}{l}
	abs a Section;\\
	 \hspace{1em}isAbstractOf article.\\
	 \hspace{3em}vs.\\
	abs a Abstract;\\ 
	 \hspace{1em}isPartOf article.
	\end{tabular}
\end{quote}

, that is, whether it is better to describe the relation between an article and its abstract component by saying that ``that section is the abstract of the article'' or that ``that section is an abstract, and it is part of the article''. Here we omit all the RDF prefixes in the statements emphasize the modeling rationale.

\subsection{Result generating process modeling}
\label{subsec:process}
Results in research publications often are quite separated from the underlying collection and analysis of data. The grand goal of keeping track of provenance is to enable the readers to understand the process the authors have gone through to produce the reported results from the collected data.

Provenance describes the lineage of the source data and the changing processes leading to the final results for readers to correctly interpret report content. Provenance also enables readers to evaluate the credibility of the reported results by digging into the software in use, source data and responsible agents.

Using general provenance ontologies such as PROV-O\footnote{PROV-O: The PROV Ontology: http://www.w3.org/TR/prov-o/}, the new W3C standard adopted in 2013, proves to be an effective way to keep track of the lineage of the source data and the changing processes leading to the final results.

In a specific domain of interest, a lot more contextual information that is commonly adopted by the community could be added to general ontologies, leading to specialized ontologies that include much more operationally meaningful elements for automated processing of the provenance graphs encoded in such ontologies.

For example, Figure~\ref{fig:gcis} shows what a typical provenance graph fragment looks like if it is created by reusing PROV-O.
\begin{figure}
	\includegraphics[width=\textwidth]{gcis-prov.png}
	\caption{Provenance graph fragment encoded in PROV-O \comment{to delete the caption inside the figure}}
	\label{fig:gcis}
\end{figure}
The example is drawn from the Global Change Information System: Information Model and Semantic Application Prototypes (GCIS-IMSAP) project\footnote{Global Change Information System: Information Model and Semantic Application Prototypes: http://tw.rpi.edu/web/project/gcis-imsap}. The project models and captures provenance information for the recent National Climate Assessment (NCA) draft report\footnote{The full draft for public review is available at http://downloads.globalchange.gov/nca/nca3-drafts/NCAJan11-2013-publicreviewdraft-fulldraft.pdf} of the US Global Change Research Program (USGCRP).

From the provenance graph, we could get the information that the paper ``paper/103'' was derived from the dataset ``dataset/103'', which in turn was derived from the dataset ``dataset/TOPEX-POSEIDON'', which was generated by the activity ``activity/.../TOPEX-POSEIDON'' that used the platform ``platform/TOPEX-POSEIDON''. Irrelevant URI parts are omitted in the graph to make the meaning of the graph clear. Also note that all PROV-O properties use past tense verbs to emphasize that everything recorded in the provenance graph must be something that already happened.

Such information is quite useful for the general public to know the process of generating the reported results in the paper. However, for domain scientists who would like to verify the reported results or to base their new research on the work reported in this paper, the provenance graph is short on critical details such as:
\begin{itemize}
	\item How the reported results in ``paper/103'' were derived from ``dataset/103'', e.g., what data items were used to create a certain plot in the paper.
	\item How ``dataset/103'' was derived from ``dataset/TOPEX-POSEIDON'', e.g., what changes were made on what part of the original dataset to generate the derived one.
	\item How ``activity/.../TOPEX-POSEIDON'' used ``platform/TOPEX-POSEIDON'' to generate ``dataset/TOPEX-POSEIDON'', e.g., whether an algorithm and/or a model was used to deal with the raw data from the platform, what parameter values were used to run the algorithm and/or model.
\end{itemize} 
We define our provenance ontology for research publications, called PROV-PUB-O, by specializing the "activity" class and the "used" property in PROV-O to make the ontology suitable for capturing executable provenance in research publications.

Note that ``wasDerivedFrom'' do not need to be specialized because ``e1 wasDerivedFrom e2'' is a shortcut for ``a1 used e1; a1 generated e2'', where e1, e2 are PROV-O entities and a1 is an PROV-O activity. Neither do ``wasGeneratedBy'' since how an activity used its input entities already contains all the information needed for replicating the generation. The property ``wasGeneratedBy'' merely points out the output entity of the activity.

Interesting activities in the process of preparing research papers are all the changes of data, which can be classified into the following three categories:
\begin{itemize}
	\item Physical changes such as data download, copying, or sharing.
	\item Syntactical changes such as XML to JSON conversion.
	\item Semantic changes such as data analysis and transformation.
\end{itemize}
Each of the above changes corresponds to a certain way of data usage.

The specialized ontology is not only helpful in describing the provenance, but it also enables the construction of executable provenance graphs to preserve the data product preparing process at a level that is detailed enough to be replicable.

\begin{figure}
	\includegraphics[width=\textwidth]{prov-pub-o.png}
	\caption{Illustrated sample provenance graph in PROV-PUB-O \comment{to replace this figure with a more developed one}}
\end{figure}

The hypothesis we would like to justify here is that sub-properties of prov:used are more useful than prov:used itself in the use case of creating executable provenance graphs in the domain of earth science.

\subsubsection{Workflow vs. provenance semantics}

\subsubsection{Executable provenance graphs}

\subsubsection{Ontology as software}

\subsection{Ontology evaluation approach}
\label{subsec:evaluation}
\comment{We need to find a recurring scientific data analytics task as our use case. Then analyze the pros and cons of using general and specialized provenance ontologies.}

\subsubsection{Objective vs. subjective ontology evaluation}

\section{Automatic provenance capturing mechanism} 
VisTrails \cite{freire2014reproducibility}
Architecture for provenance systems \cite{groth2006architecture}

Hypotheses to justify: Capturing provenance with this mechanism is more useful than capturing provenance with the workflow systems.

\comment{Compare Groth's architecture with the proposed mechanism}

\section{A proof-of-concept platform}
Hypotheses:

3: The implemented platform satisfies the requirements indicated by the mechanism.

3.1: The platform does not require explicit input from authors to capture provenance for research publications.

3.2: The platform is more useful than workflow systems for carrying out operations in research publication preparation process.

3.3: The platform captures sufficient provenance for replicating research publications.

%\resetfootnote %this command starts footnote numbering with 1 again.

%\begin{figure}
%\centering
%\vspace{2.0in}
%\caption[A Shorter Caption for the List of Figures]
%   {This is the Caption for the First Figure in Chapter 2.  It is a
%    long, long caption; we do not want to put the whole thing in the
%    List of Figures. A Shorter Caption can go in the square brackets.}
% If you like additional lines in the caption indented, see the root template
% file rpithes.tex for an example of using the caption package to do this.
%\end{figure}
 
%This is shown in table~\ref{mytable}.  % see \label below
 
%\begin{table}
%\caption{This is the Caption for Table 2}
%\label{mytable}        % \label command must always comes AFTER the caption
%\begin{center}
%\begin{tabular}{lll}
%Here's       & another     & example  \\
%of           & a           & table    \\
%floated      & with        & the      \\
%\verb+table+ & environment & command.
%\end{tabular}
%\end{center}
%\end{table}


%\section{This is a Section Heading}
 
%\subsection{This is a Subsection Heading} 
 
%Text before a footnote.\footnote{Here's the text of the footnote.}
%Text after the footnote.


%%% Local Variables: 
%%% mode: latex
%%% TeX-master: t
%%% End: 
