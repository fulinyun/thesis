%%%%%%%%%%%%%%%%%%%%%%%%%%%%%%%%%%%%%%%%%%%%%%%%%%%%%%%%%%%%%%%%%%% 
%                                                                 %
%                            ABSTRACT                             %
%                                                                 %
%%%%%%%%%%%%%%%%%%%%%%%%%%%%%%%%%%%%%%%%%%%%%%%%%%%%%%%%%%%%%%%%%%% 
 
\specialhead{ABSTRACT}

Provenance is critical for research publication readers to correctly interpret important content and enables them to evaluate the credibility of the reported results by digging into the software in use, source and change of data and responsible agents. It also would enable the reader to reproduce the scientific conclusions by following or adapting the process leading to the reported results. However, the creation of proper provenance for research publications may cost the authors a lot if they lack the necessary knowledge and technical support. First, it requires knowledge of proper logical provenance information to capture for the report creating process, causing extra learning overhead on the authors. Second, it may also require technical knowledge of the physical configurations of the program(s) execution platform such as the operating system or even the computer hardware, in order to obtain useful provenance information for the purpose of reproducibility and validation of the content. This usually entails even more learning overhead. Even if the authors already know what provenance should get recorded and how to record it, the actual recording work is usually distracting to the authors focusing on authoring the research publications and thus insufficiently motivated.

Existing frameworks and systems for capturing provenance for computational experiments are either specifically tailored for scientific workflow systems or based on a model that is not detailed enough for reproduction of the published results. Authors who are not familiar with any workflow system need to learn how to use one of these systems in order to create provenance that is detailed enough for reproducibility with them.

In this thesis, we specify a paradigm of preparing research publications based on invocation of operations to overcome many of the challenges associated with provenance capture mentioned above. The paradigm is to create publication on a portable provenance aware platform that transparently captures the proper provenance information. The PROV-PUB-O ontology is created for capturing proper knowledge provenance for authoring processes based on invocations of operations, as well as describing and locating the published results in research publications. To evaluate the usability of PROV-PUB-O, we created the Ontology Usability Scale (OUS), which is the first set of metrics for ontology usability evaluation.

The provenance capture framework enabling the paradigm that fulfills the following requirements is elaborated. First, the provenance captured must be stored in a way that the reproducibility of the reported results can be decided and the "false paths" can be found in the provenance graph that cause a certain result to not be reproducible. Second, the authoring platform must use a front end supporting a variety of programming languages/modes used by real researchers to create results. The objective is to keep the learning overhead to a minimum. Third, it is also required that the capture of provenance needs no or minimal involvement of the users. A prototype platform is implemented to demonstrate the specified framework. Chapter 4 of the 2014 U.S. National Climate Assessment report (NCA2014) is the example use case and the reproduction enabling provenance of tables and figures in this chapter is shown to be captured by the prototype.



%Provenance is critical for research publication readers to correctly interpret report content and enables them to evaluate the credibility of the reported results by digging into the software in use, source data and responsible agents. It also enables the readers to reproduce the scientific conclusions by changing the process leading to the reported results. However, creating proper provenance for research publications is very hard. First, it requires knowledge of proper provenance information to capture for the report creating process, causing either extra learning overhead on the domain scientists who write the reports or communication overhead between domain scientists and informatics experts who know how to keep track of the provenance. Second, since provenance information is usually closely related to the computational environment of the programs used during the process of preparing publications, it is often necessary to also capture the configurations of program running platform such as the operating system or even the computer hardware in order to get useful provenance information for the purpose of enabling publication readers to reproduce and validate the content, which requires knowledge of the computational system. This knowledge is usually outside the reach of both domain scientists and informatics experts, so they would depend on system administrators to get or learn by themselves the configuration related provenance information, causing more communication/learning burden. Third, it requires extra efforts to capture provenance on top of the authoring work, which already includes writing manuscripts, getting data from various sources, using various tools to process data and running programs to generate tables and figures, so the creation of provenance information is usually distracting to the authors and thus insufficiently motivated. 

%Existing scientific workflow management systems such as VisTrails \cite{freire2014reproducibility}, Kepler \cite{ludascher2006scientific}, Taverna \cite{wolstencroft2013taverna} and ReproZip \cite{chirigati2013reprozip} do not provide satisfactory solutions to any of the three problems mentioned above. Users need to learn how to use these systems in order to create provenance information with them. Information for program running platforms either is not taken care of at all, or heavily depends on the specific system architecture, e.g., by tracking operating system calls, as in the case of the ReproZip system. These systems require their users to take the extra efforts of integrating the provenance into the workflow systems, and usually authors do not have enough incentive to do this job after writing the publications.

%In this thesis, we propose a paradigm of preparing research publications to overcome all the problems associated with provenance capturing in the preparing process, which is to create publications with libraries that transparently capture the proper provenance information on a portable platform.



%The proper provenance information to be captured is defined in an ontology designed to model the preparing process of research publications. Although there are workflow models developed for capturing provenance in computational experiments \cite{groth2006architecture, groth2009recording}, models for provenance in publication preparation are not found, so even if the authors are willing to take the pains to record the necessary provenance information and make it public, they may find themselves lost in the problem of \emph{what to record and how to encode}. The authoring process modeling ontology aims to tell the authors what is necessary for the reproducibility of their publications, and the recorded information can be readily encoded with the Resource Description Framework.
%%Unlike models for computational experiments, the ontology presented in this thesis pays special attention to the human factor in the process, which is a significant feature of the publication preparing process compared with the computational experiment workflows. The provenance capturing mechanism is driven by the ontology.

%A portable platform is a software package or a Web application that runs on some environment that is available on all the popular computational systems, such as a Java program which runs on the Java Virtual Machine, a Python program which runs with the Python interpreter, or a Web based application written in HTML, JavaScript and/or PHP scripts which runs on any of the mainstream Web browsers. Preparing research publications based entirely on a portable platform eliminates the necessity of capturing provenance information down to the computer system level. Two desirable features for the platform are that 1) it should supply all the functions needed in creating research publications, and 2) these functions can be used in a way that is similar to the software packages the author is used to working with. To ensure the first feature, the platform needs to be extensible, better if it already has a strong development community to supply a plethora of evolving modules. The second feature requires that the platform supports popular interactive editing modes widely accepted in the research community. Examples include those seen in research software packages such as MATLAB and R.

%Transparent provenance capturing means the authors using the platform are not exposed to the provenance information model that drives the capturing mechanism, which is triggered each time a publication preparing action is taken, e.g., when a block of manuscript is written, when some data are downloaded from an external source, or when a plot is created by calling a library function with the downloaded data. Capturing provenance information in this way eliminates the need to ask the authors to explicitly input provenance information. It also removes from the authors the burden of learning what provenance information to capture or communicating with people who are familiar with provenance capturing.

%Contributions of the work in this thesis are 1) the creation of an ontology for capturing provenance in the process of research publication preparation; 2) the specification of a provenance capturing mechanism driven by the ontology. The mechanism makes the provenance capturing actions transparent to the authors of research publications; and 3) the implementation of a provenance aware publication preparing platform prototype that fulfills the requirements posed by the specified provenance capturing mechanism.

%We originally planned to cover all the three aspects mentioned above in this proposal, but while we investigated the first one, i.e., the ontology for capturing provenance during publication preparation, we found that the evaluation of such an ontology is itself an interesting and big research topic, so this proposal will focus on this first aspect.
%\comment{A library that transparently captures provenance information enables researchers to focus on the science part of creating publications, without being distracted by the provenance preserving part. As long as the functions provided by the library are invoked how to ensure functionalities active development community}
%
%\comment{Typical activities include importing a library to the workspace, calling a function to get source data or to create an image, adding a text block to the publication, and rerunning a code block with a different set of parameters.}
%
%\comment{The challenges, existing efforts and their shortcomings will be written here.}
%
%\comment{Feasibility comes here.}
%
%\comment{Contributions come here.}
