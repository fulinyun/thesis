%%%%%%%%%%%%%%%%%%%%%%%%%%%%%%%%%%%%%%%%%%%%%%%%%%%%%%%%%%%%%%%%%%% 
%                                                                 %
%                           BIBLIOGRAPHY                          %
%                                                                 %
%%%%%%%%%%%%%%%%%%%%%%%%%%%%%%%%%%%%%%%%%%%%%%%%%%%%%%%%%%%%%%%%%%% 
 
%This method produces a numbered bibliography where the numbers
%correspond to the \cite commands in the text. See the LaTeX manual.
%
\specialhead{REFERENCES}
\begin{singlespace}
\begin{thebibliography}{99}
\bibitem{Freire2014} Freire J., Koop D., Chirigati F., Silva C. (2014). Reproducibility using VisTrails, in Implementing Reproducible Computational Research, eds Stodden V., Leisch F., Peng R., editors. (Boca Raton, FL: Chapman \& Hall/CRC; ), (in press). Available online at: http://www.crcpress.com/product/isbn/9781466561595 
\bibitem{Ludascher2005} B. Ludascher and et. al. Scientific Workflow Management and the Kepler System. CCP\&E, 2005.
\bibitem{Taverna} Taverna. http://www.taverna.org.uk
\bibitem{Chirigati2013} Fernando Chirigati, Dennis Shasha, and Juliana Freire. Reprozip: Using prove­
nance to support computational reproducibility. In Proceedings of the 5th USENIX Workshop on the Theory and Practice of Provenance, TaPP ’13, pages 1:1–1:4, Berkeley, CA, USA, 2013. USENIX Association.

\end{thebibliography}
\end{singlespace}

% Note that, if you wish, you can use BibTeX to create your bibliography
% from a database. See section 5.6.2 of Memo RPI.110 for information. 
%%% Local Variables: 
%%% mode: latex
%%% TeX-master: t
%%% End: 
