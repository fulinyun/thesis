%%%%%%%%%%%%%%%%%%%%%%%%%%%%%%%%%%%%%%%%%%%%%%%%%%%%%%%%%%%%%%%%%%% 
%                                                                 %
%                           RELATED WORK                          %
%                                                                 %
%%%%%%%%%%%%%%%%%%%%%%%%%%%%%%%%%%%%%%%%%%%%%%%%%%%%%%%%%%%%%%%%%%% 
 
\chapter{RELATED WORK}
%\resetfootnote %this command starts footnote numbering with 1 again.

\section{General Discussions on Provenance}
\cite{jarvis2010importance} talks about the importance of provenance in the context of journalism. 

\section{Provenance Capturing Approaches}
\cite{groth2009recording} presents five characteristics of provenance information, namely immutable, meaning intact after creation, attributable, meaning clear responsibility, autonomously creatable, meaning created by the most appropriate agent, finalizable, meaning clear timing of completeness, and process reflecting, meaning able to reflect the whole process leading to the final product. It also presents the concept of p-assertions first defined in \cite{groth2006architecture} and the six key actors in provenance-aware systems, namely application, sender, receiver, asserter, recorder and provenance store.

\cite{miles2011prime} presents a provenance question driven methodology for provenance capturing.

Workflow systems such as VisTrails \cite{freire2014reproducibility}, Kepler \cite{ludascher2006scientific}, Taverna \cite{wolstencroft2013taverna} and \cite{ReproZip}.

%%% Local Variables: 
%%% mode: latex
%%% TeX-master: t
%%% End: 
