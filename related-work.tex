%%%%%%%%%%%%%%%%%%%%%%%%%%%%%%%%%%%%%%%%%%%%%%%%%%%%%%%%%%%%%%%%%%% 
%                                                                 %
%                           RELATED WORK                          %
%                                                                 %
%%%%%%%%%%%%%%%%%%%%%%%%%%%%%%%%%%%%%%%%%%%%%%%%%%%%%%%%%%%%%%%%%%% 

\chapter{RELATED WORK}
%\resetfootnote %this command starts footnote numbering with 1 again.

\section{General discussions on provenance}
\cite{jarvis2010importance} talks about the importance of provenance in the context of journalism. 
according to \cite{tan2007provenance}, the provenance information discussed in this thesis falls into the category of workflow (or coarse-grained) provenance, where detailed transformation processes of specific pieces of data in the final publications are not captured. For example, the process of generating a table is captured, but not the process that leads to specific columns, rows or cells of the table, which includes data transformation details such as the aggregation function used and the deletion of outliers.

\section{Provenance capturing approaches}
\cite{groth2009recording} presents five characteristics of provenance information, namely immutable, meaning intact after creation, attributable, meaning clear responsibility, autonomously creatable, meaning created by the most appropriate agent, finalizable, meaning clear timing of completeness, and process reflecting, meaning able to reflect the whole process leading to the final product. It also presents the concept of p-assertions first defined in \cite{groth2006architecture} and the six key actors in provenance-aware systems, namely application, sender, receiver, asserter, recorder and provenance store.

\cite{miles2011prime} presents a provenance question driven methodology for provenance capturing.

Workflow systems such as VisTrails \cite{freire2014reproducibility}, Kepler \cite{ludascher2006scientific}, Taverna \cite{wolstencroft2013taverna} and ReproZip \cite{chirigati2013reprozip}.

\section{Provenance for research publications}
\subsection{Models for publication structure}
\cite{taboada2006rhetorical} rhetorical structure theory
\cite{groza2007salt} semantically annotate \LaTeX \ source files so that roles played by each part of the publication are made explicit.
\cite{clark2013micropublications} micropublications: a semantic model for claims, evidence, arguments and annotations in biomedical communications.
\subsection{Models for Publication Preparing Process}
PROV-O

%%% Local Variables: 
%%% mode: latex
%%% TeX-master: t
%%% End: 
