%%%%%%%%%%%%%%%%%%%%%%%%%%%%%%%%%%%%%%%%%%%%%%%%%%%%%%%%%%%%%%%%%%% 
%                                                                 %
%                            CHAPTER ONE                          %
%                                                                 %
%%%%%%%%%%%%%%%%%%%%%%%%%%%%%%%%%%%%%%%%%%%%%%%%%%%%%%%%%%%%%%%%%%% 
 
\chapter{INTRODUCTION}
\section{What does provenance mean in this thesis}
According to Oxford English Dictionary, provenance is the chronology of the ownership, custody or location of a historical object. This thesis focuses on provenance for research publications, or scientific works, so in this thesis, the word \emph{provenance} means information about data, their steward activities, and agents involved in the production of results reported in research publications, as well as the linkage among such data, activities and agents. Here \emph{data} may come from observations, model runs or repositories supporting data reuse; \emph{data steward activities} include semantic, syntactical and physical changes of data; \emph{agents} are people, organizations and software agents relevant to the data and their steward activities at any phase of the data lifecycle; \emph{results} are a specialkind of data that are the final products reported in papers.

Semantic changes of data mean the changes in the actual content of data. Examples include data analysis (changing from source data to (aggregated) analysis results), data comparison (changing from two pieces of data to their comparison results), data verification (from data to their verification results) and data transformation. 

Syntactical changes of data mean the changes in serialization of data. A typical example is changing data serialized in XML to JSON without changing the actual content of the data.

Physical changes of data mean changes in locations, character encodings and accessibility of data. For example, data downloaded from an online source are the result of changing the locations of the original data (now they are in two places). When the location of a piece of textual data changes, the character encoding might change as well if the source and target storages follow different encoding schemes. Changes of data sharing permissions are a kind of physical changes that change the accessibility of data. 

\section{Why is provenance important in scientific works}
Provenance helps answer questions such as \emph{What data are the reported results based on, and where does this piece of data come from}, \emph{Have the data been modified, in what ways} \cite{davidson2008provenance}, which are critical for readers of research publications to understand and reproduce the reported results, thus they make the scientific work verifiable. 

% reproducibility, replicability, repeat, reuse
% 2014-09-29-reproducibility.txt
Reproducibility does not only mean the readers' capability to perform exactly the same data steward activities as the authors did, but also the capability of carrying out different experiments to validate the scientific work \cite{drummond2009replicability}. 




%%% Local Variables: 
%%% mode: latex
%%% TeX-master: t
%%% End: 
