%%%%%%%%%%%%%%%%%%%%%%%%%%%%%%%%%%%%%%%%%%%%%%%%%%%%%%%%%%%%%%%%%%% 
%                                                                 %
%                            CHAPTER ONE                          %
%                                                                 %
%%%%%%%%%%%%%%%%%%%%%%%%%%%%%%%%%%%%%%%%%%%%%%%%%%%%%%%%%%%%%%%%%%% 
 
\chapter{INTRODUCTION}
The scope of this thesis is limited to the provenance capturing task in the scenario of preparing research publications. It is interesting and critical to study provenance capturing in the context of computational experiments run on distributed systems, but this is not the focus of this thesis. It is assumed in this thesis that provenance information on the publication preparation process alone, not necessarily accompanied with provenance of computational experiment results, is useful to make the publications credible.

Research publication preparation has its own workflow that is very different from the extensively studied computational experiment workflows. One significant feature of the publication preparing process is that human authors are so much involved in the process that their activities are indispensable provenance information and need to be recorded. In the computational experiment settings, however, human beings are only marginally involved as operators who start the experiments and read the results, i.e., they only appear at both ends of the workflow and rarely perform as key components in the middle, so we do not see human factors discussed in existing literature, where the focus is put on software componnents forming the computational workflow.

Authors of a research publication have all the resources to record everything happened which leads to the work, but provenance pieces are scattered throughout the whole process of creating the final publication. For example, to create a tabular summary of a dataset, authors need to first use some download software package such as wget to download the dataset from a certain place online, and then use some data viewing software package to study the dataset, followed by creating the summary table with some table making software package. During this process of table creation, we would recognize more or less the following provenance pieces.
\begin{itemize}
\item What download software package was used, including its name, distribution and version.
\item Where the dataset was downloaded, including its URL.
\item When the dataset downloading activity started and ended.
\item Who executed the downloading operation.
\item The information about the data viewing software package used.
\item The time period of the dataset studying activity.
\item Who studied the dataset.
\item The information about the table making software package.
\item The time period of making the summary table.
\item Who created the table.
\end{itemize}
Figure~\cite{prov-pieces} illustrates how these pieces of provenance information scatter all over the creation process of the summary table.
\begin{figure}
\centering
\includegraphics{}
\caption{How provenance information scatters throughout a table creation process}
\end{figure}
All these pieces of information can be captured by the authors if they are informed of the list of information items they need to record and they pay sufficient attention at certain time points where certain information pieces are available, but to capture them, the authors would get distracted from their writing and table creation work quite often. Given that writing research papers is resource intensive and requires concentration, capturing provenance during writing brings no immediate benefit but lowers the productivity of the authors, so authors do not have incentives to record provenance unless they are required to do so.

\cite{jarvis2010importance}

 
\section{This is a Section Heading}
 
This is a sentence to take up space and look like text.
This is a sentence to take up space and look like text.
This is a sentence to take up space and look like text.
 
\subsection{This is a Subsection Heading} 
 
This is a sentence to take up space and look like text.
This is a sentence to take up space \cite{anotherbook}.
This is a sentence to take up space and look like text.
 
\subsubsection{This is a Subsubsection Heading}
This is a sentence to take up space and look like text.
This is a sentence to take up space and look like text.
Text before the footnote.\footnote{Here's the text of the footnote.}
Text after the footnote. 
This is a sentence to take up space and look like text.

%%% Local Variables: 
%%% mode: latex
%%% TeX-master: t
%%% End: 
