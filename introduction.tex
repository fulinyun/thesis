%%%%%%%%%%%%%%%%%%%%%%%%%%%%%%%%%%%%%%%%%%%%%%%%%%%%%%%%%%%%%%%%%%% 
%                                                                 %
%                            CHAPTER ONE                          %
%                                                                 %
%%%%%%%%%%%%%%%%%%%%%%%%%%%%%%%%%%%%%%%%%%%%%%%%%%%%%%%%%%%%%%%%%%% 
 
\chapter{INTRODUCTION}
\section{What does provenance mean in this thesis}
According to Oxford English Dictionary, provenance is the chronology of the ownership, custody or location of a historical object. This thesis focuses on provenance for research publications, or scientific works, so in this thesis, the word \emph{provenance} means information about data, their steward activities, and agents involved in the production of results reported in research publications, as well as the linkage among such data, activities and agents. Here \emph{data} may come from observations, model runs or repositories supporting data reuse; \emph{data steward activities} include semantic, syntactical and physical changes of data; \emph{agents} are people, organizations and software agents relevant to the data and their steward activities at any phase of the data lifecycle; \emph{results} are a special kind of data that are the final products reported in papers.

Semantic changes of data mean the changes in the actual content of data. Examples include data analysis (changing from source data to (aggregated) analysis results), data comparison (changing from two pieces of data to their comparison results), data verification (from data to their verification results) and data transformation. 

Syntactical changes of data mean the changes in serialization of data. A typical example is changing data serialized in XML to JSON without changing the actual content of the data.

Physical changes of data mean changes in locations, character encodings and accessibility of data. For example, data downloaded from an online source are the result of changing the locations of the original data (now they are in two places). When the location of a piece of textual data changes, the character encoding might change as well if the source and target storages follow different encoding schemes. Changes of data sharing permissions are a kind of physical changes that change the accessibility of data. 

All these changes are performed by agents, mostly software agents such as software tools, programs using library functions and command line scripts, through running software in command line, GUI or interactive mode, function invocations, and script running.

Information about people and organization agents tells the readers who owns, is responsible for, or should get credit for the publication and the data and software it used, which makes it possible to recognize and reward not only the publication authors, but also data and software producers. Recognition of these people or organizations benefits all researchers with better data and software \cite{parsons2010data, goble2014better}. \comment{better reference for data citation?}

Useful information about software agents, other than their owners, responsible parties and contributors, falls into two categories. The first one is static information such as source code and documentation. This kind of information is needed for readers of the publications to understand and reuse the software to \emph{reproduce} the scientific conclusions. The second category of software agent information is dynamic information, i.e., information associated with activities these agents perform. Examples include their running environments such as library dependencies and system environment variable values, and configurations such as parameters and command line arguments, at each time they run. Such information plays an important role in \emph{replicating} computational experiments.

Note that the two words \emph{reproduce} and \emph{replicate} are used carefully here, whose meanings will be discussed in detail in the next section.


\section{Why is provenance important in scientific works}
With richer and richer provenance information, a piece of scientific work would get the following desirable features in order, namely \emph{transparency}, \emph{replicability} and \emph{reproducibility}. At the very basic level, provenance helps answer questions such as \emph{``What data are the reported results based on, and where does this piece of data come from''}, \emph{``Have the data been modified, in what ways''} \cite{davidson2008provenance}, which makes a piece of scientific work \emph{transparent}, meaning the readers have access to the knowledge of the data, their steward activities and associated agents. With richer provenance on data steward activities and software agents, readers have the ability to perform the same experiments carried out by the authors, making the work \emph{replicable}, meaning the same experiment can be carried out in a different lab, according to Goble's keynote presentation at ISMB/ECCB 2013. The final goal of provenance is to enable readers to carry out different experiments to validate the same scientific conclusion that the original experiment tries to justify, which is, according to \cite{drummond2009replicability}, \emph{reproducibility}.

The first feature, transparency, not only increases the trustworthiness of the scientific work by making the research process leading to the publication open to public scrutiny, but also reveals the work that would otherwise not be recognized, such as the development, configuration, integration and deployment of software for experiments \cite{goble2014better}.

Here \emph{replicability} and \emph{reproducibility} are defined as in \cite{drummond2009replicability}. In the last decade, the word \emph{reproducibility} pops up a lot in discussions about provenance. For example, Altintas et al., in \cite{altintas2004kepler}, claims that scientific research is generally held to be of good provenance when it is documented in detail sufficient to allow reproducibility, and Boose et al., in \cite{boose2007ensuring}, pointed out that data sets are reliable when the process used to create them are reproducible and analyzable for defects. \comment{These are sentences copied from Wikipedia, need to fix citations.} Drummond pointed out in \cite{drummond2009replicability} that \emph{reproducibility} in the sense of carrying out the same experiments by different researchers should actually be called \emph{replicability}. 


Reproducibility does not only mean the readers' capability to perform exactly the same data steward activities as the authors did, but also the capability of carrying out different experiments to validate the scientific work \cite{drummond2009replicability}. 
\cite{goble2014better}
\comment{read 2014-10-20-computational-vs-scientific-reproducibility.txt and 2014-09-29-reproducibility.txt}

\comment{read 2014-09-30-introduction-writing-order.txt}


%%% Local Variables: 
%%% mode: latex
%%% TeX-master: t
%%% End: 
