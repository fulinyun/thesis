\chapter{FUTURE WORK}
\label{future-work}
\section{Workflow vs. provenance semantics}
Provenance is always about something that indeed \emph{happened}, i.e., not about something that \emph{is happening}, {\emph{will happen} or \emph{may happen}. Workflow is different, however, that it may contain conditional branches to express control logics such as \emph{if \dots then \dots else \dots} Therefore, it is important that we keep our mind clear on the semantic differences between provenance and workflow. Especially for provenance ontologies aiming at describing the details of the workflow leading to the data product in question. We plan to perform detailed checks on semantics of the proposed provenance ontology.

\section{Executable provenance graphs}
One reason we need to pay special attention to the semantic subtlety between provenance and workflow is that the proposed ontology will be able to define provenance graphs that are detailed enough to \emph{run} just like workflows. It is envisioned that such executable provenance graphs would be driven by standard RDF/OWL reasoners. We believe this approach to reproducibility has advantages over existing workflow systems in terms of implementability, maintainability and extensibility. We will compare the two approaches in detail in the final thesis.

\section{Ontology as software}
The comparison between workflow systems and executable provenance graphs inspire us that there might be more ways to ``put part of software into ontologies'', i.e., to transfer logics that would otherwise written with programming languages to ontologies. In this sense, ontologies could be seen as transparent software with logics explicitly and formally described. We hope we could find interesting insights on this topic in the final thesis.

\section{Automatic provenance capturing mechanism}
This topic is originally a task with the same weight as ``creating a provenance capturing ontology''. But the ontology topic expands unexpectedly well and we decided to focus on it and put the provenance capturing mechanism as a future work. The mechanism is designed to make provenance capturing chores less distracting for the authors. As mentioned in Section \ref{sec:reproducibility}, Donoho et al admitted that recording provenance hurts productivity, although in the long run its benefits would way supersede its shortcomings. We believe it is possible to overcome the short-term counter-productive weakness of working reproducibly by using authoring tools that transparently record provenance for the authors. \comment{As mentioned in...}

\comment{Compare Groth's architecture with the proposed mechanism}

\section{A proof-of-concept platform}
Hypotheses:

3: The implemented platform satisfies the requirements indicated by the mechanism.

3.1: The platform does not require explicit input from authors to capture provenance for research publications.

3.2: The platform is more useful than workflow systems for carrying out operations in research publication preparation process.

3.3: The platform captures sufficient provenance for replicating research publications.


%%% Local Variables: 
%%% mode: latex
%%% TeX-master: t
%%% End: 
