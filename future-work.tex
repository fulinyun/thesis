\chapter{CONCLUSION AND FUTURE WORK}
\label{future-work}

\section{Conclusion}
In this thesis, we specified a solution to the problems faced by researchers who would like to present their research results in a reproducible way but lack the knowledge of and/or cannot afford the distraction introduced by manual provenance recording practices. A provenance ontology suitable for describing and locating research results, as well as capturing provenance of result generation processes based on operation invocations was presented to encode the knowledge of proper provenance that should be captured. The ontology was evaluated in terms of usability and was shown to be comparable with state-of-the-art application ontologies for provenance capture. A provenance aware framework was specified to deal with provenance capture for the common paradigm of generating published results based on operation invocations. The effectiveness of the framework was evaluated through an investigation that showed the framework could capture provenance of all the tables and figures from a chapter in the NCA 2014 report. Finally, a prototype compliant with the framework specification was implemented as a proof of concept.

\section{Building a community of provenance aware research software developers}
Although it has been shown in Section~\ref{sec:framework} that the provenance capture framework accommodates sufficient functions for research result generation, and sample Python modules for data and result generation have been implemented and described in Section~\ref{sec:prototype}, it requires the implementation of more provenance aware functions to make the prototype usable for researchers. For example, libraries frequently used by researchers to work with data such as NumPy and SciPy in Python should be extended to become provenance aware in order to make the prototype usable. The challenge in this work is that developers of such provenance aware libraries need to be knowledgeable in both the research data processing software and the provenance model. We believe in an optimistic future because the demand for provenance aware research software keeps increasing, in turn this will foster more and more generally usable provenance models and/or models usable for specific domains to emerge, and it is possible that there will be one provenance model that is widely adopted by research software developers and/or models that is widely adopted in certain research domains, and once the community of \emph{provenance aware developers} expands, and a certain amount of research software compliant with the models is developed, we suggest (without providing evidence) that the models may be even more widely adopted by both ontology users and domain scientists because of the availability of existing software libraries to interact with the ontologies. We see a desirable future where a mature provenance aware platform is widely adopted by researchers. Such a platform begins with a group of highly usable provenance models, so that usability of the provenance model plays an important role in the community building process.

\section{Specializations of PROV-O for different research domains}
As pointed out in Section~\ref{sec:prov-pub-eval}, the PROV-PUB-O ontology may be biased towards the Earth and space science domain due to the background knowledge of its developers. It is possible that each domain has an ideal model that is adopted by most researchers but that all these models do not converge. We see this situation as an advance over the one where each workflow system has its own internal provenance representation scheme, because a domain specific provenance model does not need to be interpreted when used by different provenance aware platforms within the same domain, given that the model is supported by both platforms. Whereas in the situation of each system having its own provenance representation, provenance information is system specific and at least some syntactical translation is needed to enable interactions across different systems. To create a widely adopted domain specific provenance models, both domain scientists and provenance experts need to get involved, so the challenge lies in facilitating effective communication between the two groups of people so that ideas widely adopted among domain scientists are encoded in highly usable ontologies.

\section{Ontology usability evaluation approaches}
The Ontology Usability Scale (OUS) presented in Section~\ref{subsec:evaluation} is a first attempt to create a set of metrics that could measure the general usability of an ontology. It borrows some of the ideas of System Usability Scale (SUS) presented by Brooke in \cite{brooke1996sus}, and chooses the statements composing the scale based on a semiotic framework presented by Burton-Jones et al. in \cite{burton2005semiotic}. There is no usability expert involved in the development of OUS, so better approaches to evaluate the usability of ontologies are likely to emerge when usability experts, ontology experts and domain scientists work together to develop such approaches.

Evaluating and comparing ontologies developed for the same domain or application are still not common practices, partly due to the small number of domain and application ontologies available. Demand for evaluation approaches on all aspects of ontologies is likely to increase as ontologies are more widely adopted as data, information and knowledge models and ontology users have more choices for their interested domains and applications.

\section{Prototype usability evaluation}
In this thesis we just validated the effectiveness of the provenance capturing framework and the prototype implemented based on it. Since provenance aware functions replace the normal functions used by researchers when the prototype is used to create research results, and researchers are required to perform all operations with the prototype to generate results, the usability of the prototype is worth evaluating. The challenge here is to distinguish between usability issues caused by lack of functions and those caused by having to use a different set of functions. General usability metrics such as SUS do not help in telling such differences, but we are interested in the latter kind of usability issues to improve the usage paradigm (that is, usability) rather than just implementing more functions (that is, usefulness). 
%\section{Workflow vs. provenance semantics}
%Provenance is always about something that indeed \emph{happened}, i.e., not about something that \emph{is happening}, {\emph{will happen} or \emph{may happen}. Workflow is different, however, that it may contain conditional branches to express control logics such as \emph{if \dots then \dots else \dots} Therefore, it is important that we keep our mind clear on the semantic differences between provenance and workflow. Especially for provenance ontologies aiming at describing the details of the workflow leading to the data product in question. We plan to perform detailed checks on semantics of PROV-PUB-O.

%\section{Executable provenance graphs}
%One reason we need to pay special attention to the semantic subtlety between provenance and workflow is that PROV-PUB-O will be able to define provenance graphs that are detailed enough to \emph{run} just like workflows. It is envisioned that such executable provenance graphs would be driven by standard RDF/OWL reasoners. We believe this approach to reproducibility has advantages over existing workflow systems in terms of implementability, maintainability and extensibility. We will compare the two approaches in detail in the final thesis.

%\section{Ontology as software}
%The comparison between workflow systems and executable provenance graphs inspire us that there might be more ways to ``put part of software into ontologies'', i.e., to transfer logics that would otherwise written with programming languages to ontologies. In this sense, ontologies could be seen as transparent software with logics explicitly and formally described. We hope we could find interesting insights on this topic in the final thesis.

%\section{Automatic provenance capturing mechanism}
%This topic is originally a task with the same weight as ``creating a provenance capturing ontology''. But the ontology topic expands unexpectedly well and we decided to focus on it and put the provenance capturing mechanism as a future work. The mechanism is designed to make provenance capturing chores less distracting for the authors. As mentioned in Section \ref{sec:reproducibility}, Donoho et al admitted that recording provenance hurts productivity, although in the long run its benefits would way supersede its shortcomings. We believe it is possible to overcome the short-term counter-productive weakness of working reproducibly by using authoring tools that transparently record provenance for the authors. As mentioned in Section \ref{sec:possibility}, we will try to develop provenance aware tools with the same or similar user interfaces with existing tools to save learning time for researchers, and these tools save provenance in the background so they do not require explicit effort of recording provenance from the authors.

%\comment{Compare Groth's architecture with the proposed mechanism}

%\section{A proof-of-concept platform}
%A natural follow-up of the automatic provenance capturing mechanism is to develop a proof-of-concept platform that demonstrates the desired features of provenance aware authoring tools. As mentioned in Section \ref{sec:contribution}, we are looking for a portable and function-rich implementation with user-familiar interfaces.

%Hypotheses:

%3: The implemented platform satisfies the requirements indicated by the mechanism.

%3.1: The platform does not require explicit input from authors to capture provenance for research publications.

%3.2: The platform is more useful than workflow systems for carrying out operations in research publication preparation process.

%3.3: The platform captures sufficient provenance for replicating research publications.


%%% Local Variables: 
%%% mode: latex
%%% TeX-master: t
%%% End: 
