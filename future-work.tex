\chapter{FUTURE WORK}
\label{future-work}
\section{Workflow vs. provenance semantics}

\section{Executable provenance graphs}

\section{Ontology as software}

\section{Automatic provenance capturing mechanism} 
VisTrails \cite{freire2014reproducibility}
Architecture for provenance systems \cite{groth2006architecture}

Hypotheses to justify: Capturing provenance with this mechanism is more useful than capturing provenance with the workflow systems.

\comment{Compare Groth's architecture with the proposed mechanism}

\section{A proof-of-concept platform}
Hypotheses:

3: The implemented platform satisfies the requirements indicated by the mechanism.

3.1: The platform does not require explicit input from authors to capture provenance for research publications.

3.2: The platform is more useful than workflow systems for carrying out operations in research publication preparation process.

3.3: The platform captures sufficient provenance for replicating research publications.


%%% Local Variables: 
%%% mode: latex
%%% TeX-master: t
%%% End: 
